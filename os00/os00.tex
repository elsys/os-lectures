\documentclass[ignorenonframetext, hyperref=unicode]{beamer}



\usepackage{cmap}
%\usepackage[T2A]{fontenc}
\usepackage[utf8]{inputenc}
\usepackage[bulgarian]{babel}
\selectlanguage{bulgarian}

\usepackage{color}
\usepackage{graphicx}
\usepackage{listings}
\usepackage{rcsinfo}
\usepackage{pgf}
\usepackage{supertabular}
\usepackage{rotating}

\hypersetup{
	colorlinks=true,
	linkcolor=blue,
	filecolor=blue,
	urlcolor=blue,
	anchorcolor=blue,
	citecolor=blue
}

\lstset{language=C++, 
  numbers=left, 
  numberstyle=\tiny,
  stepnumber=1, 
  numbersep=3pt, 
  tabsize=2, 
  texcl,
  basicstyle=\ttfamily\small,
  identifierstyle=\ttfamily\small,
  keywordstyle=\sffamily\bfseries\small,
  extendedchars=true, inputencoding=utf8,
  backgroundcolor=\color[rgb]{1,1,0.845},
  escapeinside={/*@}{@*/}}

%\usepackage{algpseudocode}
%\usepackage[ruled]{algorithm}

\newcommand{\Cpp}{{\ttfamily\bfseries C++}}
\newcommand{\CC}{{\ttfamily\bfseries C}}

\definecolor{outputcolor}{rgb}{0.0,0.0,0.5}
\newcommand{\aout}[1]{\color{outputcolor}{\begin{verbatim}#1\end{verbatim}}}

% \usepackage[T2A]{fontenc}
% \usepackage[cp1251]{inputenc}
% \usepackage[bulgarian]{babel}
\selectlanguage{bulgarian}




\newcommand{\lubo}{%
\author[Л.~Чорбаджиев]{Любомир Чорбаджиев\inst{1} \\ 
{\ttfamily lchorbadjiev@elsys-bg.org}}
\institute[ELSYS] % (optional, but mostly needed)
{
\inst{1}%
Технологическо училище ``Електронни системи'' \\
Технически университет, София
}}

\newcommand{\osauthors}{%
\author{
	В.Кетипов\\ 
	\and
	Н.Димитров \\ 
	\and
	{Х.Стефанов \\
	{\ttfamily elsys.os.2014@gmail.com}}
}
\institute[ELSYS] % (optional, but mostly needed)
{
\inst{1}%
Технологическо училище ``Електронни системи'' \\
Технически университет, София
}}

\titlegraphic{\href{http://creativecommons.org/licenses/by-sa/3.0/}{\includegraphics{../macros/cc.png}}}

\newcommand{\ie}{т.~е.\ }

\newcounter{probcounter}[section]
\newenvironment{prob}[1][]%
        {\smallskip%
         \noindent\refstepcounter{probcounter}%
          \textbf{\theprobcounter${}^{#1}$.}\ }%
   {\medskip}

\mode<article>
{

}

\mode<presentation>
{
  \usetheme[secheader=true]{Madrid}
  \usecolortheme{crane}
  \usefonttheme[onlylarge]{structurebold}
  \setbeamercovered{transparent}
}

\usepackage[unicode]{hyperref}

%%% Local Variables: 
%%% mode: latex
%%% TeX-master: t
%%% End: 



\title{Организация на курса \\ ``Операционни системи"} \lubo
\date{\today}

\begin{document}

\frame{\maketitle}

\begin{frame}
\frametitle{Съдържание}
\tableofcontents %[hideallsubsections]
\end{frame}

\section{Организация на курса}

%---------------------------------------------------------------------- SLIDE -
\begin{frame}
\frametitle{Организация на курса}
\begin{itemize}
\item Два семестъра.
\begin{itemize}
  \item Първи семестър: теория -- 2 часа седмично.
  \item Втори семестър: практика -- 2 часа седмично.
\end{itemize}
\item Две оценки:
\begin{itemize}
  \item първия семестър -- по теория;
  \item втория семестър -- по практика.
\end{itemize}
\end{itemize}
\end{frame}

\section{Учебници}
%---------------------------------------------------------------------- SLIDE -
\begin{frame}
\frametitle{Учебници}

\begin{itemize}
\item 
Abraham Silberschatz and Peter Baer Galvin and Greg Gagne.
{\em Operating System Concepts}.
{John Wiley \& Sons, Inc., 6rd edition, 2001.}
\item 
Andrew S. Tanenbaum.
{\em Modern Operating Systems}.
{Prentice Hall PTR, 2nd edition, 2001.}
\end{itemize}

\end{frame}

\section{Ресурси}
%---------------------------------------------------------------------- SLIDE -
\begin{frame}
\frametitle{Ресурси}
\begin{itemize}
\item Страница на курса: \url{http://lubo.elsys-bg.org}.
\item За въпроси и коментари: \url{lchorbadjiev@elsys-bg.org}
\end{itemize}
\end{frame}

\end{document}
\end{document}