\documentclass[ignorenonframetext, hyperref=unicode]{beamer}



\usepackage{cmap}
%\usepackage[T2A]{fontenc}
\usepackage[utf8]{inputenc}
\usepackage[bulgarian]{babel}
\selectlanguage{bulgarian}

\usepackage{color}
\usepackage{graphicx}
\usepackage{listings}
\usepackage{rcsinfo}
\usepackage{pgf}
\usepackage{supertabular}
\usepackage{rotating}

\hypersetup{
	colorlinks=true,
	linkcolor=blue,
	filecolor=blue,
	urlcolor=blue,
	anchorcolor=blue,
	citecolor=blue
}

\lstset{language=C++, 
  numbers=left, 
  numberstyle=\tiny,
  stepnumber=1, 
  numbersep=3pt, 
  tabsize=2, 
  texcl,
  basicstyle=\ttfamily\small,
  identifierstyle=\ttfamily\small,
  keywordstyle=\sffamily\bfseries\small,
  extendedchars=true, inputencoding=utf8,
  backgroundcolor=\color[rgb]{1,1,0.845},
  escapeinside={/*@}{@*/}}

%\usepackage{algpseudocode}
%\usepackage[ruled]{algorithm}

\newcommand{\Cpp}{{\ttfamily\bfseries C++}}
\newcommand{\CC}{{\ttfamily\bfseries C}}

\definecolor{outputcolor}{rgb}{0.0,0.0,0.5}
\newcommand{\aout}[1]{\color{outputcolor}{\begin{verbatim}#1\end{verbatim}}}

% \usepackage[T2A]{fontenc}
% \usepackage[cp1251]{inputenc}
% \usepackage[bulgarian]{babel}
\selectlanguage{bulgarian}




\newcommand{\lubo}{%
\author[Л.~Чорбаджиев]{Любомир Чорбаджиев\inst{1} \\ 
{\ttfamily lchorbadjiev@elsys-bg.org}}
\institute[ELSYS] % (optional, but mostly needed)
{
\inst{1}%
Технологическо училище ``Електронни системи'' \\
Технически университет, София
}}

\newcommand{\osauthors}{%
\author{
	В.Кетипов\\ 
	\and
	Н.Димитров \\ 
	\and
	{Х.Стефанов \\
	{\ttfamily elsys.os.2014@gmail.com}}
}
\institute[ELSYS] % (optional, but mostly needed)
{
\inst{1}%
Технологическо училище ``Електронни системи'' \\
Технически университет, София
}}

\titlegraphic{\href{http://creativecommons.org/licenses/by-sa/3.0/}{\includegraphics{../macros/cc.png}}}

\newcommand{\ie}{т.~е.\ }

\newcounter{probcounter}[section]
\newenvironment{prob}[1][]%
        {\smallskip%
         \noindent\refstepcounter{probcounter}%
          \textbf{\theprobcounter${}^{#1}$.}\ }%
   {\medskip}

\mode<article>
{

}

\mode<presentation>
{
  \usetheme[secheader=true]{Madrid}
  \usecolortheme{crane}
  \usefonttheme[onlylarge]{structurebold}
  \setbeamercovered{transparent}
}

\usepackage[unicode]{hyperref}

%%% Local Variables: 
%%% mode: latex
%%% TeX-master: t
%%% End: 


\title[Указатели и динамична памет]{Указатели и динамична памет} 
\osauthors
\date{\today}

\begin{document}

\frame{\maketitle}

\begin{frame}
\frametitle{Съдържание}
\tableofcontents %[hideallsubsections]
\end{frame}

%-------------------------------------------------------------------- SECTION -
\section{Въведение}

%---------------------------------------------------------------------- SLIDE -
\begin{frame}
\frametitle{Въведение}
\begin{itemize}
  \item Компютърна паметта представлява проста последователност от байтове.
  \item По никакъв начин не е обозначено къде започват едни данни и къде свършват.
  \item Няма означение и за техния тип, предназначение или име.
\end{itemize}
\end{frame}

%---------------------------------------------------------------------- SLIDE -
\begin{frame}
\frametitle{Въведение}
\begin{itemize}
  \item Когато една програма на C бъде компилирана, тя се трансформира от код на езика C в машинен код.
  \item Машинният код представлява поредица от инструкции, които процесорът разбира.
  \item В този машинен код за обръщение към данни от паметта се използват само и единствено адреси.
\end{itemize}
\end{frame}

%---------------------------------------------------------------------- SLIDE -
\begin{frame}
\frametitle{Адрес}
\begin{itemize}
  \item Адресът в паметта представлява отместване от началото й.
  \item Тъй като рядко ни е необходим само един бит от паметта, е въведено по-голямото понятие байт.
  \item Отместването се измерва в байтове. Тоест адрес 100 означава, че данните се намира на 100 байта след началото на паметта.
\end{itemize}
\end{frame}

%---------------------------------------------------------------------- SLIDE -
\begin{frame}
\frametitle{Байт}
\begin{itemize}
  \item След като адресите са в байтове, то ние не можем да подадем инструкция към процесора, която да прочете само един бит
  \item Поради тази причина определeнието за байт е най-малката адресируема единица, т.е. най-малкото количество данни, с които процесорът може да работи.
  \item В почти всички случаи един байт е 8 бита, което е избрано, тъй като може да представи един символ.
  \item За работа с определени битове от даден байт трябва да използваме побитови операции на процесора, които променят данните бит по бит, но резултатът от тях винаги е байт.
\end{itemize}
\end{frame}


%-------------------------------------------------------------------- SECTION -
\section{Адреси и указатели}


%---------------------------------------------------------------------- SLIDE -
\begin{frame}[containsverbatim]
\frametitle{Адреси в C}
\begin{itemize}
  \item В езикът C можем да видим адресът на данните, които използваме.
  \item Това се постига чрез оператора \&
  \item Адресите обикновено се извеждат в шестнайсетична бройна система.
\end{itemize}
\begin{lstlisting}
	int a = 42;
	printf("%p\n", (void *) &a); //0x7fff02bf4bac
\end{lstlisting}

\end{frame}

%---------------------------------------------------------------------- SLIDE -
\begin{frame}[containsverbatim]
\frametitle{Тип на указател}
\begin{itemize}
  \item Тъй като указателя е просто отместване, ние трябва да укажем колко байтa искаме да прочетем
  \item Поради това указателите имат тип - int*, char*, long*, double* и други.
  \item В езикът C има и указател без тип - void*. За да прочетем нещо от указател void*, трябва да го превърнем към указател от някой друг тип.
\end{itemize}
\begin{lstlisting}
	int a = 42;
	void *p = &a;
	int b = *p; // Грешка
	int c = *(int *)p; //42
\end{lstlisting}
\end{frame}


%-------------------------------------------------------------------- SECTION -
\section{Динамична памет}

%---------------------------------------------------------------------- SLIDE -
\begin{frame}[containsverbatim]
\frametitle{Локални променливи}
\begin{itemize}
  \item В езикът C част от паметта се управлява автоматично - заделя се при влизане в дадена функция и се освобождава при излизането от нея.
\end{itemize}
\begin{lstlisting}
	void func1() {
		int a = 5;
		int b = 6;
	}
	
	void func2() {
		int k;
		func1();
	}
	
	int main() {
		func2();
	}

\end{lstlisting}
\end{frame}

\begin{frame}[containsverbatim]
\frametitle{Динамична памет}
\begin{itemize}
  \item Възможно е и ние да управляваме ръчно паметта, което се налага в множество случаи в C
  \item За да използваме някакъв участък от паметта е необходимо да поискаме памет от операционната система.
  \item Това се налага, тъй като ОС се грижи да разпределя паметта между различните процеси.
\end{itemize}
\end{frame}

\begin{frame}[containsverbatim]
\frametitle{malloc}
\begin{itemize}
  \item За да заделим памет се използва системната функция malloc, която приема като аргумент броя байтове, които искаме да заделим
  \item malloc върща указател към заделената памет
  \item Ако malloc върне NULL, то това означава грешка. В повечето случаи причината е липсата на памет.
\end{itemize}
\begin{lstlisting}
	char *p = malloc(1); //Заделя 1 байт
\end{lstlisting}
\end{frame}

\begin{frame}[containsverbatim]
\frametitle{sizeof}
\begin{itemize}
  \item Важно е да се подава правилния брой байтове към malloc
  \item За целта се използва оператора sizeof
  \item sizeof връща колко байта са необходими за даден тип
\end{itemize}
\begin{lstlisting}
	struct point {
		double x;
		double y;
	} 

	//Заделя памет за 1 int
	int *p1 = malloc(sizeof(int)); 
	
	 //Заделя памет за 2 double
	struct point *p2 = malloc(sizeof(struct point));
	
\end{lstlisting}
\end{frame}

\begin{frame}[containsverbatim]
\frametitle{sizeof}
\begin{itemize}
  \item sizeof(char *) НЕ е равно на sizeof(char)
  \item sizeof(char) връща броя байтове, необходими за запомнянето на един char
  \item sizeof(char*) връща броя байтове, необходими за запомнянето на един указател
  \item Всички указатели са една и съща големина, определена от архитектурата на процесора (32 или 64 бита)
  \item sizeof(char*) = sizeof(int*) = sizeof(double*)
\end{itemize}
\end{frame}

\begin{frame}[containsverbatim]
\frametitle{sizeof}
\begin{itemize}
	\item За да запомним един адрес винаги са ни необходими един и същи брой байтове, независимо от типа.
	\item Типът определя какво седи на този адрес, а не големината на самия адрес.
	\item Адресът просто показва къде започват данните в паметта и поради това големината му не зависи от това какво се намира на този адрес.
\end{itemize}
\begin{lstlisting}
	//Големината в байтове е примерна
	char a = 'X'; //1 байт
	char *pa = &a; //4 байта(при 32 битови процесори)
	
	int b = 42; //4 байта
	int *pb = &b; //4 байта
	
\end{lstlisting}
\end{frame}

\begin{frame}[containsverbatim]
\frametitle{Масиви}
\begin{itemize}
	\item Заделянето на масив динамично е аналогично с заделянето на една променлива.
	\item Единствената разлика е, че е необходимо да заделим памет за всеки елемент.
\end{itemize}
\begin{lstlisting}
	//Големината в байтове е примерна
	
	//Масив от 100 елемента, 400 байта
	int *arr = malloc(100 * sizeof(int));
	
	//Масив от 100 елемента, 100 байта
	char *arr = malloc(100 * sizeof(char)); 
	
\end{lstlisting}
\end{frame}

\begin{frame}[containsverbatim]
\frametitle{Адресна аритметика}
\begin{itemize}
	\item Възможно е да добавяме и изваждаме числа от указател.
	\item Така можем да достъпваме следващите елементи от масив
	\item Операторът arr[i] е просто по-удобен запис за *(arr + i);
\end{itemize}
\begin{lstlisting}
	//Масив от 100 елемента, 400 байта
	int *arr = malloc(100 * sizeof(int));
	arr[0] = 5; //Първият елемент
	arr[1] = 6; //Вторият елемент
	
	printf("%d\n", *(arr + 1)); //Извежда 6
	
	arr++;
	printf("%d\n", arr[0]); //Извежда 6
	printf("%d\n", arr[-1]); //Извежда 5
	
\end{lstlisting}
\end{frame}

\begin{frame}[containsverbatim]
\frametitle{Адресна аритметика}
\begin{itemize}
	\item Нека приемем, че имаме масив от int елемента, започващи от адрес 0x100
	\item Ако приемем, че за един int са необходими 4 байта, то втория елемент ще намира на адрес 0x104
	\item Нека приемем, че имаме масив от char елемента, започващи от адрес 0x500
	\item Ако приемем, че за един char е необходим 1 байт, то втория елемент ще намира на адрес 0x501
\end{itemize}
\end{frame}

\begin{frame}[containsverbatim]
\frametitle{Адресна аритметика}
\begin{itemize}
	\item Операторите +, -, [] взимат предвид тези особености
	\item Очевидно не можем да инкрементираме void*, тъй като не знаем големината на елемента, към който сочи.
\end{itemize}
\begin{lstlisting}
	int *arr = malloc(100 * sizeof(int));
	printf("%p\n", (void *) arr); //0x100
	arr++;
	printf("%p\n", (void *) arr); //0x104
	
	char *carr = malloc(100 * sizeof(char));
	printf("%p\n", (void *) carr); //0x500
	carr++;
	printf("%p\n", (void *) carr); //0x501
	
\end{lstlisting}
\end{frame}

\begin{frame}[containsverbatim]
\frametitle{Освобождаване на паметта}
\begin{itemize}
	\item За да укажем на ОС, че вече не ни е необходима даден памет е необходимо да я освободим.
	\item Това става чрез функцията free.
	\item Освобождаването на памет трябва да става възможно най-скоро след като вече не ни е необходима, за да могат други програми да я използват.
\end{itemize}
\begin{lstlisting}
	int *arr = malloc(100 * sizeof(int));
	....
	....
	free(arr);
\end{lstlisting}
\end{frame}

\end{document}
