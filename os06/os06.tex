\documentclass[ignorenonframetext, hyperref=unicode]{beamer}



\usepackage{cmap}
%\usepackage[T2A]{fontenc}
\usepackage[utf8]{inputenc}
\usepackage[bulgarian]{babel}
\selectlanguage{bulgarian}

\usepackage{color}
\usepackage{graphicx}
\usepackage{listings}
\usepackage{rcsinfo}
\usepackage{pgf}
\usepackage{supertabular}
\usepackage{rotating}

\hypersetup{
	colorlinks=true,
	linkcolor=blue,
	filecolor=blue,
	urlcolor=blue,
	anchorcolor=blue,
	citecolor=blue
}

\lstset{language=C++, 
  numbers=left, 
  numberstyle=\tiny,
  stepnumber=1, 
  numbersep=3pt, 
  tabsize=2, 
  texcl,
  basicstyle=\ttfamily\small,
  identifierstyle=\ttfamily\small,
  keywordstyle=\sffamily\bfseries\small,
  extendedchars=true, inputencoding=utf8,
  backgroundcolor=\color[rgb]{1,1,0.845},
  escapeinside={/*@}{@*/}}

%\usepackage{algpseudocode}
%\usepackage[ruled]{algorithm}

\newcommand{\Cpp}{{\ttfamily\bfseries C++}}
\newcommand{\CC}{{\ttfamily\bfseries C}}

\definecolor{outputcolor}{rgb}{0.0,0.0,0.5}
\newcommand{\aout}[1]{\color{outputcolor}{\begin{verbatim}#1\end{verbatim}}}

% \usepackage[T2A]{fontenc}
% \usepackage[cp1251]{inputenc}
% \usepackage[bulgarian]{babel}
\selectlanguage{bulgarian}




\newcommand{\lubo}{%
\author[Л.~Чорбаджиев]{Любомир Чорбаджиев\inst{1} \\ 
{\ttfamily lchorbadjiev@elsys-bg.org}}
\institute[ELSYS] % (optional, but mostly needed)
{
\inst{1}%
Технологическо училище ``Електронни системи'' \\
Технически университет, София
}}

\newcommand{\osauthors}{%
\author{
	В.Кетипов\\ 
	\and
	Н.Димитров \\ 
	\and
	{Х.Стефанов \\
	{\ttfamily elsys.os.2014@gmail.com}}
}
\institute[ELSYS] % (optional, but mostly needed)
{
\inst{1}%
Технологическо училище ``Електронни системи'' \\
Технически университет, София
}}

\titlegraphic{\href{http://creativecommons.org/licenses/by-sa/3.0/}{\includegraphics{../macros/cc.png}}}

\newcommand{\ie}{т.~е.\ }

\newcounter{probcounter}[section]
\newenvironment{prob}[1][]%
        {\smallskip%
         \noindent\refstepcounter{probcounter}%
          \textbf{\theprobcounter${}^{#1}$.}\ }%
   {\medskip}

\mode<article>
{

}

\mode<presentation>
{
  \usetheme[secheader=true]{Madrid}
  \usecolortheme{crane}
  \usefonttheme[onlylarge]{structurebold}
  \setbeamercovered{transparent}
}

\usepackage[unicode]{hyperref}

%%% Local Variables: 
%%% mode: latex
%%% TeX-master: t
%%% End: 



\title[Синхронизация {(\em\rcsInfoRevision)}]{Синхронизация между процеси} 
\lubo
\date{\today}

\begin{document}

\frame{\maketitle}

\begin{frame}
\frametitle{Съдържание}
\tableofcontents %[hideallsubsections]
\end{frame}

%-------------------------------------------------------------------- SECTION -
\section{Производител/потребител}

%---------------------------------------------------------------------- SLIDE -
\begin{frame}[containsverbatim]
\frametitle{Производител/потребител}
\begin{lstlisting}
#define BUFFER_SIZE 10
typedef struct {
	//...
} item_t;
item_t buffer[BUFFER_SIZE];
int in=0;
int out=0;
\end{lstlisting}
\end{frame}

%---------------------------------------------------------------------- SLIDE -
\begin{frame}
\frametitle{Производител/потребител}
\begin{figure}[h]
\center
\scalebox{0.4}{\includegraphics{pics/stallings4e-finite-circular-buffer}}
\end{figure}
\end{frame}
%---------------------------------------------------------------------- SLIDE -
\begin{frame}[containsverbatim]
\frametitle{Производител}
\begin{lstlisting}
	item_t nextProduced;

	while (1) {
		...
		/* produce item in nextProduced */
		...
		while (((in + 1) % BUFFER_SIZE) == out)
			; /* do nothing */
		buffer[in] = nextProduced;
		in = (in + 1) % BUFFER_SIZE;
	}
\end{lstlisting}
\end{frame}


%---------------------------------------------------------------------- SLIDE -
\begin{frame}[containsverbatim]
\frametitle{Потребител}
\begin{lstlisting}
	item_t nextConsumed;

	while (1) {
		while (in == out)
			; /* do nothing */
		nextConsumed = buffer[out];
		out = (out + 1) % BUFFER_SIZE;
		...
		/* consume item from nextConsumed */		
		...
	}
\end{lstlisting}
\end{frame}

%---------------------------------------------------------------------- SLIDE -
\begin{frame}[containsverbatim]
\frametitle{Производител/потребител}
\begin{itemize}
  \item В общите променливи добавяме нова променлива \lstinline{counter}, която
  брои колко обекта има в момента в буфера \lstinline{buffer}. 
  \item Ако  \lstinline{counter==0}, то буфера е празен.
  \item Ако  \lstinline{counter==BUFFER_SIZE}, то буфера е пълен.
\end{itemize}
\begin{lstlisting}
#define BUFFER_SIZE 10
typedef struct {
	//...
} item_t;
item_t buffer[BUFFER_SIZE];
int in=0;
int out=0;
int counter=0;
\end{lstlisting}
\end{frame}

%---------------------------------------------------------------------- SLIDE -
\begin{frame}[containsverbatim]
\frametitle{Производител}
\begin{lstlisting}
	item_t nextProduced;

	while (1) {
		...
		/* produce item in nextProduced */
		...
		while (counter==BUFFER_SIZE)
			; /* do nothing */
		buffer[in] = nextProduced;
		in = (in + 1) % BUFFER_SIZE;
		counter++;
	}
\end{lstlisting}
\end{frame}


%---------------------------------------------------------------------- SLIDE -
\begin{frame}[containsverbatim]
\frametitle{Потребител}
\begin{lstlisting}
	item_t nextConsumed;

	while (1) {
		while (counter==0)
			; /* do nothing */
		nextConsumed = buffer[out];
		out = (out + 1) % BUFFER_SIZE;
		counter--;
		...
		/* consume item from nextConsumed */		
		...
	}
\end{lstlisting}
\end{frame}


%-------------------------------------------------------------------- SECTION -
\section{Условие на надпревара (race condition)}

%---------------------------------------------------------------------- SLIDE -
\begin{frame}[containsverbatim]
\frametitle{Атомарност}
\begin{itemize}
  \item За да бъде коректно предложеното решение на проблема
  производител/потребител, е необходимо операциите:
  \begin{lstlisting}[numbers=none]
	counter++;
	counter--;
  \end{lstlisting}
  да бъдат атомарни.
  \item {\em Атомарна} се нарича операция, която се изпълнява изцяло, без да бъде
  прекъсвана от операционната система.
\end{itemize}
\end{frame}

%---------------------------------------------------------------------- SLIDE -
\begin{frame}[containsverbatim]
\frametitle{Атомарност}
\begin{itemize}
  \item Операциите
  \begin{lstlisting}[numbers=none]
	counter++;
	counter--;
  \end{lstlisting}
  типично не са атомарни.
  \item Операцията \lstinline{counter++} на ниво процесор се изпълнява
  по следния начин:
\begin{lstlisting}
	register1=counter;
	register1=register1+1;
	counter=register1;
\end{lstlisting}
  \item Операцията \lstinline{counter--} на ниво процесор се изпълнява
  по следния начин:
\begin{lstlisting}
	register2=counter;
	register2=register2-1;
	counter=register2;
\end{lstlisting}
\end{itemize}
\end{frame}

%---------------------------------------------------------------------- SLIDE -
\begin{frame}[containsverbatim]
\frametitle{Условие на надпревара}
\begin{itemize}
  \item Когато и производителя и потребителя се опитат конкурентно да променят
  състоянието на буфера, инструкциите им могат да се преплетат.
  \item Преплитането на инструкциите на производителя и потребителя зависи от
  начина, по който двата процеса се планират върху процесора.
\end{itemize}
\end{frame}

%---------------------------------------------------------------------- SLIDE -
\begin{frame}[containsverbatim]
\frametitle{Условие на надпревара}
\begin{itemize}
  \item Да предположим, че първоначално \lstinline{counter=5}. Един вариант за
  преплитане на производителя и потребителя е следния:
\begin{lstlisting}
producer: register1=counter     /*register1 = 5*/
producer: register1=register1+1 /*register1 = 6*/
consumer: register2=counter     /*register2 = 5*/
consumer: register2=register2-1 /*register2 = 4*/
producer: counter=register1     /*counter = 6  */
consumer: counter=register2     /*counter = 4  */
\end{lstlisting}
  \item Стойността на \lstinline{counter} може да бъде 4 или 6, докато правилния
  резултат трябва да бъде 5.
\end{itemize}
\end{frame}

%---------------------------------------------------------------------- SLIDE -
\begin{frame}[containsverbatim]
\frametitle{Условие на надпревара}
\begin{itemize}
  \item {\em Условие на надпревара (race condition)} се нарича ситуацията, при която
  няколко
  процеса конкурентно манипулират данни, които са общи, поделени между конкурентните
  процеси.
  \item Крайната стойност на поделените (общите) данни зависи от начина, по
  който процесите се планират върху процесора.
  \item За да се предотврати състоянието на надпревара (race condition),
  конкурентните процеси трябва да бъдат синхронизирани.
\end{itemize}
\end{frame}

%-------------------------------------------------------------------- SECTION -
\section{Критичен регион (critical section)}

%---------------------------------------------------------------------- SLIDE -
\begin{frame}[containsverbatim]
\frametitle{Критичен секция (critical section)}
\begin{itemize}
  \item Няколко процеса се състезават за достъп до общи (поделени) данни.
  \item Всеки процес има участък от кода, в който работи с общите за всички процеси
  данни. Такъв участък от кода се нарича {\em критична секция}.
  \item Трябва да се изгради механизъм, чрез който да се гарантира, че когато
  един процес се намира в критична секция, никой друг процес не може да
  влезе в своята критична секция.
\end{itemize}
\end{frame}



%---------------------------------------------------------------------- SLIDE -
\begin{frame}[containsverbatim]
\frametitle{Критичен секция (critical section)}
Този механизъм трябва да притежава следните свойства:
  \begin{itemize}
    \item Взаимно изключване: когато процесът $P_i$ се намира в критична секция,
    никой друг процес не може да навлезе в своята критична секция. Във всеки
    един момент от време само един процес може да се намира в критична секция.
    \item Процес, който не се намира в критична секция, не може да влияе на
    другите процеси.
    \item Не трябва да се забавя влизането на процес в критична секция, ако в
    момента няма друг процес в критична секция.
    \item Механизмът не трябва да разчита на предположения относно скоростта на
    изпълнение на процесите.
  \end{itemize}
\end{frame}

%---------------------------------------------------------------------- SLIDE -
\begin{frame}[containsverbatim]
\frametitle{Критичен секция (critical section)}
Общата структура на процеси, притежаващи критична секция, е следната:
\begin{lstlisting}
do {
	// вход в критичната секция;
	...
	// критична секция;
	...
	// изход от критичната секция;
	...
	// некритична секция;
	...
	...
} while(...);
\end{lstlisting}
\end{frame}

%-------------------------------------------------------------------- SECTION -
\section{Алгоритми за синхронизация}

%---------------------------------------------------------------------- SLIDE -
\begin{frame}[containsverbatim]
\frametitle{Алгоритми за синхронизация}
\begin{itemize}
  \item Нека разгледаме два процеса -- $P_0$ и $P_1$. Нека с $i$ обозначим
  номера на единия процес, тогава номера на другия процес ще бъде $j=1-i$.
  \item Нека двата процеса имат следната обща променлива, която служи за
  синхронизация между процесите:
\begin{lstlisting}[numbers=none]
int turn=0;
\end{lstlisting}
  \item Стойността на променливата \lstinline{turn} определя кой процес може да
  влезе в критичната си секция. 
  \item Ако \lstinline{turn=0}, то процес $P_0$ може да
  влезе в критичната си секция. 
  \item Ако \lstinline{turn=1}, то процес $P_1$ може да
  влезе в критичната си секция.
\end{itemize}
\end{frame}

%---------------------------------------------------------------------- SLIDE -
\begin{frame}[containsverbatim]
\frametitle{Алгоритми за синхронизация}
Примерна структура на процеса $P_i$:
\begin{lstlisting}
do {
	while(turn!=i)
		; // do nothing
	...
	// критична секция
	...
	turn=j;
	...
	// некритична секция
	...
} while(...);
\end{lstlisting}
\end{frame}

%---------------------------------------------------------------------- SLIDE -
\begin{frame}[containsverbatim]
\frametitle{Алгоритми за синхронизация}
\begin{itemize}
\item Алгоритъмът удовлетворява условието за взаимно изключване на процесите.
Когато единия процес $P_i$ се намира в критична секция, другия процес $P_j$
изчаква, докато дойде неговия ред.
\item Коректността на решението се основава на факта, че операцията присвояване
на стойност \lstinline{turn=j;} е атомарна операция.
\item Това решение, обаче не удовлетворява останалите изисквания към механизма
за синхронизация. 
\item Когато процесът $P_i$ мине през критичната си секция, променливата
\lstinline{turn} става равна на $j$. Ако процесът $P_i$ се опита отново да влезе
в критичната си секция, то той ще бъде блокиран, защото \lstinline{turn!=i}.
Процесът $P_i$ трябва да изчака процеса $P_j$, макар че $P_j$ не се намира в
критична секция.
\end{itemize}
\end{frame}


%---------------------------------------------------------------------- SLIDE -
\begin{frame}[containsverbatim]
\frametitle{Алгоритъм на Петерсон}
\begin{itemize}
\item Нека отново разгледаме два процеса -- $P_0$ и $P_1$. Нека с $i$ обозначим
  номера на единия процес, а с $j=1-i$ номера на другия процес.
\item Нека двата процеса имат следните общи променливи, които се изполозват за
  синхронизация между двата процеса: 
\begin{lstlisting}[numbers=none]
int turn=0;
bool flag[2];
flag[0]=false;
flag[1]=false;
\end{lstlisting}
\item Стойността на променливата \lstinline{turn} определя кой процес е на ред
  да влезе в критичната си секция.
\item Стойностите на елементите в масива \lstinline{flag[]}  показват дали
  съответния процес е готов да влезе в критичната си секция. Когато
  \lstinline{flag[i]==true}, процесът $P_i$ е готов да влезе в критичната си секция.
\end{itemize}
\end{frame}

%---------------------------------------------------------------------- SLIDE -
\begin{frame}[containsverbatim]
\frametitle{Алгоритъм на Петерсон}
Примерна структура на процеса $P_i$:
\begin{lstlisting}
do {
	flag[i]=true;
	turn=j;
	while(flag[j]&& turn==j)
		; // do nothing
	...
	// критична секция
	...
	flag[i]=false;
	...
	// некритична секция
	...
} while(...);
\end{lstlisting}
\end{frame}


%-------------------------------------------------------------------- SECTION -
\section{Хардуерна поддръжка}

\subsection{Забрана на прекъсванията}

%---------------------------------------------------------------------- SLIDE -
\begin{frame}[containsverbatim]
\frametitle{Забрана на прекъсванията}
\begin{itemize}
\item Прекъсванията диктуват превключването на контекста на процесора. Когато
прекъсванията са забранени, няма какво да предизвика превключване на контекста
на процесора. 
\item Забраната на прекъсванията гарантира взаимното изключване, тъй като това
забранява превключването на контекста на процесора и прави невъзможно
преплитането на два процеса.
\item Този подход работи върху еднопроцесорни системи. Върху многопроцесорни
системи забраната на прекъсванията върху единия процесор не гарантира взаимно
изключване. 
\end{itemize}
\end{frame}

\subsection{Атомарни инструкции}

%---------------------------------------------------------------------- SLIDE -
\begin{frame}[containsverbatim]
\frametitle{Атомарни инструкции}
\begin{itemize}
\item  Съвременните процесорни архитектури предоставят атомарни инструкции --
инструкции, които гарантирано се изпълняват без да бъдат прекъсвани.
\item Тези операции се опират на факта, че на хардуерно ниво обръщането към
клетка от паметта забранява всякакви други операции с тази клетка.
\item Идеята е за един цикъл на изпълнение да се изпълнят две
операции -- четене и запис, или четене и проверка на стойността.
\item Тъй като тези операции се изпълняват за един цикъл (една инструкция), то
върху тях не могат да повлияят никакви други инструкции.
\end{itemize}
\end{frame}

\subsection{Test And Set}
%---------------------------------------------------------------------- SLIDE -
\begin{frame}[containsverbatim]
\frametitle{Test And Set}

\begin{lstlisting}
bool testandset(int* v) {
	if(*v!=0) {
		return false;
	} else {
		*v=1;
		return true;
	}
}
\end{lstlisting}
\end{frame}

%---------------------------------------------------------------------- SLIDE -
\begin{frame}[containsverbatim]
\frametitle{Test And Set}
\begin{itemize}
\item Нека разгледаме два процеса -- $P_0$ и $P_1$. Нека с $i$ обозначим
  номера на единия процес, а с $j=1-i$ номера на другия процес.
\item Нека двата процеса имат следната обща променлива, която се използва за
синхронизация между двата процеса: 
\begin{lstlisting}[numbers=none]
bool lock=false;
\end{lstlisting}
\item Стойността на променливата \lstinline{lock} определя дали има процес в
критична секция. Ако \lstinline{lock==true}, то това означава, че някой от
процесите в момента е в критична секция, и другия процес трябва да чака -- не
може да влезе в критичната си секция.
\end{itemize}
\end{frame}


%---------------------------------------------------------------------- SLIDE -
\begin{frame}[containsverbatim]
\frametitle{Test And Set}
Примерна структура на процеса $P_i$:
\begin{columns}
\column{0.5\textwidth}{%
\begin{lstlisting}[numbers=none]
do {
	while(!testandset(&lock))
		; // do nothing
	...
	// критична секция
	...
	lock=false;
	...
	// некритична секция
	...
} while(...);
\end{lstlisting}}
\column{0.5\textwidth}{%
\begin{lstlisting}[numbers=none]
bool testandset(int* v){
	if(*v!=0) {
		return false;
	} else {
		*v=1;
		return true;
	}
}
\end{lstlisting}}
\end{columns}
\end{frame}


\subsection{Swap}

%---------------------------------------------------------------------- SLIDE -
\begin{frame}[containsverbatim]
\frametitle{Swap}
\begin{lstlisting}
void swap(int* r, int* m) {
	int temp=*m;
	*m=*r;
	*r=temp;
}
\end{lstlisting}
\end{frame}


%---------------------------------------------------------------------- SLIDE -
\begin{frame}[containsverbatim]
\frametitle{Swap}
\begin{itemize}
\item Нека разгледаме два процеса -- $P_0$ и $P_1$. Нека с $i$ обозначим
  номера на единия процес, а с $j=1-i$ номера на другия процес.
\item Нека двата процеса имат следната обща променлива, която се използва за
синхронизация между двата процеса: 
\begin{lstlisting}[numbers=none]
bool lock=false;
\end{lstlisting}
\item Стойността на променливата \lstinline{lock} определя дали има процес в
критична секция. Ако \lstinline{lock==true}, то това означава, че някой от
процесите в момента е в критична секция, и другия процес трябва да чака -- не
може да влезе в критичната си секция.
\end{itemize}
\end{frame}


%---------------------------------------------------------------------- SLIDE -
\begin{frame}[containsverbatim]
\frametitle{Swap}
Примерна структура на процеса $P_i$:
\begin{columns}
\column{0.45\textwidth}{%
\begin{lstlisting}[numbers=none]
bool val=false;
do {
	val=true;
	while(val==true)
		swap(lock,val); 
	...
	// критична секция
	...
	lock=false;
	...
	// некритична секция
	...
} while(...);
\end{lstlisting}}
\column{0.55\textwidth}{%
\begin{lstlisting}[numbers=none]
void swap(int* m1, int* m2){
	int temp=*m1;
	*m1=*m2;
	*m2=temp;
}
\end{lstlisting}}
\end{columns}
\end{frame}

%---------------------------------------------------------------------- SLIDE -
\begin{frame}[containsverbatim]
\frametitle{Хардуерна поддръжка}
\begin{itemize}
\item Основно предимство на разгледаните средства за синхронизация е че те са
приложими към произволен брой процеси, които работят върху произволен брой
процесори, които си поделят обща памет (uniform memory access).
\item Разгледаните механизми са прости и могат лесно да се провери тяхната
коректност.
\item Могат да бъдат използвани за поддръжка на множество критични
секции.
\item {\em Основният недостатък} на всички разгледани механизми за синхронизация
е, че използват ``работно чакане'' (busy waiting). Докато даден процес чака своя ред
за влизане в критична секция, той не спира да работи и да използва процесорно
време.
\end{itemize}
\end{frame}


%-------------------------------------------------------------------- SECTION -
\section{Семафори}

%---------------------------------------------------------------------- SLIDE -
\begin{frame}[containsverbatim]
\frametitle{Семафори}
\begin{itemize}
\item Семафорът е специален вид променлива, която се използва за комуникация
между процесите. Използвайки семафори, даден процес може да изпраща
сигнал на друг процес и/или да очакват да получи сигнал от друг процес.
\item Когато даден процес очаква да получи сигнал, той преминава в състояние
``блокиран'' (blocked).
\item С всеки семафор е асоциирана целочислена променлива. Върху семафорите
могат да се изпълняват само две операции, които са атомарни (неделими).
\end{itemize}
\end{frame}

%---------------------------------------------------------------------- SLIDE -
\begin{frame}[containsverbatim]
\frametitle{Семафори}
\begin{itemize}
\item Примерна реализация на семафор може да се изгради върху следната
структура:
\begin{lstlisting}
struct Semaphore_t {
	int value;
	QueueType_t blocked;
} 
\end{lstlisting}
\item Членът \lstinline{value} е целочислената стойност, асоциирана със
семафора.
\item Членът \lstinline{blocked_queue} съдържа опашката от процеси, които са
блокирани върху семафора.
\end{itemize}
\end{frame}

%---------------------------------------------------------------------- SLIDE -
\begin{frame}[containsverbatim]
\frametitle{Семафори}
\begin{itemize}
\item Върху семафорите могат да се изпълняват само две операции —
\lstinline{wait()} и \lstinline{signal()}.
\item При създаването си семафорът може да бъде инициализиран с неотрицателно
цяло число.
\item Операцията \lstinline{wait()} намалява стойността на семафора с единица.
Ако стойността на семафора стане отрицателна, процеса изпълняващ операцията
блокира.
\item Операцията \lstinline{signal()} увеличава стойността на семафора с
единица. Ако върху семафора има блокирани процеси, то операцията
\lstinline{signal()} води до разблокиране на някой от блокираните процеси.
\end{itemize}
\end{frame}


%---------------------------------------------------------------------- SLIDE -
\begin{frame}[containsverbatim]
\frametitle{Семафори}
\begin{itemize}
\item Примерна реализация на операцията \lstinline{wait()}:
\begin{lstlisting}
void wait(Semaphore_t* s) {
	s->value--;
	if(s->value<0) {
		// добави процеса към s->blocked
		// блокирай процеса
	}
}
\end{lstlisting}
\end{itemize}
\end{frame}

%---------------------------------------------------------------------- SLIDE -
\begin{frame}[containsverbatim]
\frametitle{Семафори}
\begin{itemize}
\item Примерна реализация на операцията \lstinline{signal()}:
\begin{lstlisting}
void signal(Semaphore_t* s) {
	s->value++;
	if(s->value<=0) {
		// извади един процес P от s->blocked
		// постави процеса P в опашката на готовите процеси
	}
}
\end{lstlisting}
\end{itemize}
\end{frame}

%---------------------------------------------------------------------- SLIDE -
\begin{frame}[containsverbatim]
\frametitle{Семафори}
\begin{itemize}
\item Операциите \lstinline{wait()} и \lstinline{signal()} са атомарни.
\item Реализацията на механизма на семафорите изисква поддръжка от операционната
система.
\item Често като специален случай се разглежда така наречения 
{\em бинарен семафор} или {\em мутекс}. Това е семафор, чиято стойност може да
приема само две стойности -- 0 и 1.
\end{itemize}
\end{frame}


%---------------------------------------------------------------------- SLIDE -
\begin{frame}[containsverbatim]
\frametitle{Бинарен семафор}
\begin{lstlisting}
struct BinSemaphore_t {
	enum {zero,one} value;
	QueueType_t blocked;
} 
void wait(BinSemaphote_t* s) {
	if(s->value==one) {
		s->value=zero;
	} else {
		// добави процеса към s->blocked;
		// блокирай процеса
	}
}
\end{lstlisting}
\end{frame}

%---------------------------------------------------------------------- SLIDE -
\begin{frame}[containsverbatim]
\frametitle{Бинарен семафор}
\begin{lstlisting}
void wait(BinSemaphote_t* s) {
	if(s->blocked is empty) {
		s->value=one;
	} else {
		// извади един процес P от s->blocked;
		// постави процеса P в опашката на готовите процеси;
	}
}
\end{lstlisting}
\end{frame}

%---------------------------------------------------------------------- SLIDE -
\begin{frame}[containsverbatim]
\frametitle{Синхронизация със семафори}
\begin{itemize}
\item Нека разгледаме два процеса -- $P_0$ и $P_1$. Нека с $i$ обозначим
  номера на единия процес, а с $j=1-i$ номера на другия процес.
\item Нека двата процеса имат общ семафор:
\begin{lstlisting}[numbers=none]
Semaphore_t sem=//създаден и инициализиран с 1;
\end{lstlisting}
\item Когато даден процес иска да влезе в критичната си секция, той вика
операцията \lstinline{wait()} върху общия семафор. Когата процеса напуска
критичната си секция вика операцията \lstinline{signal()}.
\end{itemize}
\end{frame}


%---------------------------------------------------------------------- SLIDE -
\begin{frame}[containsverbatim]
\frametitle{Синхронизация със семафори}
Примерна структура на процеса $P_i$:
\begin{lstlisting}[numbers=none]
do {
	wait(&sem); 
	...
	// критична секция
	...
	signal(&sem);
	...
	// некритична секция
	...
} while(...);
\end{lstlisting}
\end{frame}



\end{document}
